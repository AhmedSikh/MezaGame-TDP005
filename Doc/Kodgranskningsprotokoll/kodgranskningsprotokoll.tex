\documentclass{TDP005mall}



\newcommand{\version}{Version 1.1}
\author{Alicia Bergman \url {alibe500@student.liu.se}\\
  Josefin Bodin \url {josbo110@student.liu.se}\\
  Ahmed Sikh \url {ahmsi881@student.liu.se}}
\title{Kodgranskningsprotokoll}
\date{2020-12-09}
\rhead{Alicia Bergman\\
  Josefin Bodin\\
  Ahmed Sikh}



\begin{document}
\projectpage
\section{Revisionshistorik}
\begin{table}[!h]
\begin{tabularx}{\linewidth}{|l|X|l|}
\hline
Ver. & Revisionsbeskrivning & Datum \\\hline
1.0 & Första utkast & 201209 \\\hline
\end{tabularx}
\end{table}


\section{Möte}
Inför mötet såg vi till att vår motgrupp hade tillgång till vår projektkod och
vår designspecifikation via gitlab. Vi hade även tillgång till deras på samma
sätt. Vi förberedde oss genom att först kolla igenom koden på egen hand för att
sedan diskutera tillsammans. Mötet ägde rum den 9:e december 14:00

\section{Det granskade projektet}
Spelet vi granskade går ut på att skjuta fiender som rör sig i en rad mot
spelaren. Spelarens rörelse är begränsad till en specifik area. Över denna area
rör sig en annan typ av fiende slumpmässigt och spelaren måste se till att
undvika denna eller skjuta den. 

\subsection{Första intrycket}
Vid en första anblick tyckte vi att spelets designspecifikation och kod såg mycket genomtänkt och välplanerad ut och detta var även det allmänna intrycket
när projektet lästs igenom.

\subsection{Kommentarer}
Koden var generellt sett ganska lättförståelig, men det skulle inte skada med
mer kommentarer i koden för att göra läsbarheten lite lättare och koden mer
förståelig vid en första anblick.

\subsection{Funktionsplats}
Något vi noterade var att en del funktioner som hanterade ungefär samma sak låg
i olika klasser i projektet och det vore bra om det istället låg samlade.

\subsection{Publika variabler}
Alla variabler i objectklassen var publika, något som kan skapa problem i
framtiden.

\subsection{Nya nivåer}
Vi märkte att allt för tillfället är hårdkodat, vilket gör det komplicerat om
man vill lägga till nya nivåer till spelet.

\subsection{Positiv feedback}

\subsubsection{Välplanerat}
Projektet kändes som sagt genomtänkt och välplanerat.

\subsubsection{Main}
Main funktionen är väldigt liten, vilket är ett av kraven på koden.

\subsubsection{Klasser}
De har bra och tydliga klasser som gör precis det de ska göra.

\section{Vårat projekt}

\subsection{Kollision med väggar}
Vår spelare vibrerar när den kolliderar med väggar, vår motgrupp hade haft
liknande problem och erbjöd en lösning på detta.

\subsection{Game}
Vår updatefunktion i gameklassen var svårläslig, något vi ska kolla igenom och
försöka lösa.

\subsection{State}
Vi saknar för tillfället en stateklass som alla staten ärver ifrån. Detta är
ingen prioritet för tillfället men ska försöka lösas om tid finns.

\subsection{Enemy}
Movefunktionen i enemyklassen är ganska svårläslig, vi ska försöka förbättra
läsbarheten i den.
Enemy tar för tillfället in ganska många parametrar, något som kanske kan lösas
med en vektor.

\subsection{Inläsning av sprite}
Vid inläsning av en sprite får vi nu ett errormeddelande om texturen ej kan
läsas in, detta kan lösas med hjälp av en defaulttextur istället.

\subsection{variabelnamn}
Vi har ganska korta och oförståeliga variabelnamn, tanken är att gå igenom koden
och ändra på dessa till mer passande namn.

\subsection{Positiv feedback}

\subsubsection{Generellt}
Vår motgrupp tyckte vårat projekt var välplanerat och genomtänk samt att vår
polymorfi var bra.

\subsubsection{Inläsning av nivåer}
Vi har en bra inläsning, vilket gör det lätt att läsa in nya nivåer.
\end{document}
